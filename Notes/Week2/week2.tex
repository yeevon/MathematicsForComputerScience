\documentclass[12pt]{report}
\usepackage{amssymb}
\usepackage{amsmath}

\title{\Huge{Mathematics for Computer Science} Week 2 Notes}
\author{\huge{Jose Miguel de Lima}}
\date{11/24/2025}

\begin{document}
\maketitle
\section*{Lecture Notes}

\textbf{Indirect proof or proof by contradiction:}
    To prove a proposition $P$ is true, we assume P is false | (i.e., $\neg P$ is true) and then use
    hypothesis to derive a falsehood or contradiction.

\smallskip

\noindent if $\neg P \Rightarrow F$ is true, then $P$ must be true or $\neg P$ is false.

\hrulefill
\smallskip

\noindent \textbf{Example:} Prove: $\sqrt{2}$ is irrational.

\noindent Proof by contradiction:
    Assume for the purpose of contradiction that $\sqrt{2}$ is rational.
    
    \smallskip
    \noindent$\Rightarrow\sqrt{2} = a/b$ (fraction in lowest terms)
    
    $\Rightarrow 2b^2 = a^2$
    
    $\Rightarrow a^2$ is even ($2|a$)
    
    $\Rightarrow a$ is even (since the square of an odd number is odd)
    
    $\Rightarrow 4|a^2 \Rightarrow 4|2b^2 \Rightarrow 2|b^2 \Rightarrow b$ is even
    

    $\Rightarrow a$ and $b$ are both even, which contradicts the assumption that $a/b$ is in lowest terms.
    \smallskip
    \noindent Therefore, $\sqrt{2}$ is irrational.

\hrulefill\bigskip

\noindent \textbf{Induction axiom:}
    Let $P(n)$ be a predicate, if $P(0)$ is true, and $\forall n \in \mathbb{N}$ $(P(n) \Rightarrow P(n+1))$ is true,
    then $\forall n \in \mathbb{N}, P(n)$ is true.

\smallskip
\noindent if $P(0)$ is true (base case) and $P(n) \Rightarrow P(n+1)$ is true (inductive step),
    then $P(n)$ is true for all $n \in \mathbb{N}$.

\hrulefill\smallskip
\newline
\noindent \textbf{Example:} Prove: $\forall n \geq 0,$ $1 + 2 + 3 + ... + n = \frac{n(n+1)}{2}$.
\noindent Other ways to write this:
    \[
        \sum_{i=1}^{n} i = \frac{n(n+1)}{2}
    \],
    \[
        \sum_{i=0}^{n} i = \frac{n(n+1)}{2}
    \] (since adding 0 doesn't change the sum),
    \[
        \sum_{i=1}^{n} i = \frac{n^2 + n}{2}
    \]

\medskip
\noindent \textbf{Special case:} 

If $n=1$ $1 + 2 + \cdots + n = 1$ (base case)

If $n \leq 0$ $1 + 2 + \cdots + n = 0$ (base case)


\bigskip
\noindent \textbf{Example:}
\[
    n = 4:\quad
    1 + 2 + 3 + 4 = 10 = \frac{4(4+1)}{2} = \frac{20}{2} = 10
\]


\hrulefill\newline
\bigskip
Proof by induction:
\noindent Let $P(n)$ be the predicate 

\[\sum_{i=1}^{n} i = \frac{n(n+1)}{2}\]

\noindent\textbf{Base Case:} $P(0)$ is true
\[
    \sum_{i=1}^{0} i = 0 = \frac{0(0+1)}{2} = 0
\]

\noindent\textbf{Inductive Step:} For $n \geq 0$, assume $P(n) \Rightarrow P(n+1)$ is true, i.e.,

Assume $P(n)$ is true for purposes of induction. 

(i.e., assume $1+2+\cdots+n = \frac{n(n+1)}{2}$)

need to show $1+2+\cdots+(n+1) = \frac{(n+1)(n+2)}{2}$

\begin{align*}
    1 + 2 + \cdots + (n+1)
        &= (1 + 2 + \cdots + n) + (n+1) \\
        &= \frac{n(n+1)}{2} + (n+1) \quad \text{(by inductive hypothesis)} \\
        &= \frac{n(n+1) + 2(n+1)}{2} \\
        &= \frac{(n+1)(n+2)}{2}
\end{align*}


\section*{Definitions}
\noindent\textbf{$...$:} figure out the pattern and complete the definition. \medskip
\end{document}