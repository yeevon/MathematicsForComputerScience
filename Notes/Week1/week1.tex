\documentclass[12pt]{report}
\usepackage{amssymb}

\title{\Huge{Mathematics for Computer Science} Week 1 Notes}
\author{\huge{Jose Miguel de Lima}}
\date{11/19/2025}

\begin{document}
\maketitle

\section*{Reading}
\noindent Chpater 1: Sections 1 - 3

\section*{Definitions}

\noindent\textbf{Proposition:} A proposition is a statement (communication) that is either true or false, but not both.
\medskip

\noindent\textbf{Predicate:} a predicate can be understood as a proposition whose truth depends on the value of one more
or more variables.
\medskip

\noindent\textbf{Axioms:} a statement or proposition which is regarded as being established, accepted, or self-evidently true.
\medskip

\noindent\textbf{Proofs:} A proof is a sequence of logical deductions from axioms and previously proved statemtns that concludes with the proposition in question.
\medskip

\noindent\textbf{Theorem:} A theorem is a proposition that has been or needs to be proved on the basis of previously established statements,
\medskip

\noindent\textbf{Lemmas:} A lemma is a proposition that is proved for use in the proof of another proposition.
\medskip

\noindent\textbf{Corollary:} A corollary is a proposition that follows in just a few logical steps from a theorem.
\medskip

\section*{Math Symbols}
\begin{itemize}
    \item \textbf{::=  :} means ``equals by definition''. 
          It's always ok to write \texttt{=} instead of \texttt{::=}.
    \item \textbf{$\forall$  :} is read ``for all'' or ``for every''.
    \item \textbf{$\mathbb{N}$  :} is the set of natural numbers: $\{0, 1, 2, 3, \ldots\}$.
    \item \textbf{$\in$  :} means ``is an element of'' or ``belongs to'' or simply as ``is in''.
    \item \textbf{$\mathbb{Z^+}$  :} is the set of positive integers: $\{1, 2, 3, \ldots\}$.
    \item \textbf{$P(n)$  :} is a predicate with variable $n$. ``note: $P(n)$ is not a proposition nor a function.''
    \item \textbf{$\exists$  :} is read ``there exists'' or ``there is at least one''.
    \item \textbf{$\wedge$  :} is read ``and''.
    \item \textbf{$\vee$  :} is read ``or''.
    \item \textbf{$\neg$  :} is read ``not''.
    \item \textbf{$\Rightarrow$  :} is read ``implies
\end{itemize}

\end{document}