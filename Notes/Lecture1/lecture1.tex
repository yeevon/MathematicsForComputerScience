\documentclass[12pt]{report}
\usepackage{amssymb}
\usepackage{amsmath}

\title{\Huge{Mathematics for Computer Science} Lecture 1 Notes}
\author{\huge{Jose Miguel de Lima}}
\date{11/25/2025}

\begin{document}
\maketitle

\section*{Lecture Notes:}

What is a Proof? : 

-- Its a method of ascertaining truth.

---------------------------------------------------

Examples: 

            - Experiments

            - Sampling 

            - Legal proccedings
            
            - Investigations
            
            - Questions
            
            - Authority
            
            - Religion
            
            - Inner Conviction
            
--------------------------------------

Def: A mathematical proof is a verification of a proposition by a chain of logical deductions from a base set of axioms.            

\medskip
Def: A proposition is a statement that is true or false.

---------------------

Examples: 

        - 2 + 3 = 5
        
        - 2 + 3 = 4

        - $\forall n \in \mathbb{N}: $ $n^2 + n + 41$ is a prime.

        - $P$ is prime (not a propostion) this is a predicate

----------------------------

Def: A predicate is a propostion whose truth depends on variables

---------------------------------

Goldbach's Conjecture: Every even number > 2 is the sum of 2 primes.

Example: 
        - 12 = 7 + 5

----------------------------------------

Def: Conjecture not sure if it true or not

---------------------------------

A, B are propositions

not A  : $\neg A$ : means a is false

A and B : $A \land B$

A or B : $A \lor B$

exclusive Or, XOR : $A \oplus B: $ One or the Other, not neither or both

Not And, NAND: $ A \barwedge B $ : $ \neg (A \land B): $ One or the Other or Neither not both

-------------------------------------

Implication: A implies B  or  A $\rightarrow$ B: ``A implies B'' aka ``if A then B''


$A \implies B$ 

 - counterpositive $\neg B \implies \neg A$
 
 - converse $B \implies A$

 - inverse $\neg A \implies \neg B$

 the converse and inverse are counterpositive of each other.

-------------------------------------------------

Def: A set is a collection of objects. Order doesn't matter and no repeats.

Example:

        - ${6, 1, 2, 0} ::= \{2. 1. 6. 0\} ::= \{6, 1, 2, 0, 0\}$

        - $\mathbb{N}  ::= \{0, 1, 2, 3, ...\}$

        - $\mathbb{Z} ::= \{0, 1, 2, 3, -1, -2, -3, ...\}$

        - $\mathbb{Q} ::= rations$

        - $\mathbb{R} ::= reals$

        - $\mathbb{\not C} ::= complex$

        - Empty Set $\not 0, $ \{\}

        - $\mathbb{B} ::= \{2, {3, 4}, \not 0\}$

------------------------

$x \in A$ means $x$ is an element of $A$

$A \subset B$ ``A is a subset or equal to B''

$A \cup B$ : ``union''

$A \cap B$ : ``intersection''

$A - B$ : ``set difference''

----------------------------------------

Set Builder Notation: 

    ${n \in \mathbb{N} \mid isprime(n)} = \{2, 3, 5, 7, 11, ...\}$

\bigskip
Tuples: ordered list of elements, and repeats are ok, uses () instead \{\}

$(6,1, 2, 0) \not = (6, 1, 2, 0, 0) \not = (2, 1, 6, 0)$

----------------------------------------

Def: an Axiom is a proposition we assume is true.

Examples:

        - Euclid's Parallel Postulate: For every point p and line l with $p \not\in l$, $\exists$ a unique line $l'$ through $p$ parallel to $l$

--------------------------

A set of Axioms is consistent when you can't prove that False is True.

... is complete when every true proposition can be proved from the axioms.

The Godel incompletness therom: Can't have both. There are true statements that cannot be proved.



\end{document}